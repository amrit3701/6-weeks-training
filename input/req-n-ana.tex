\section{Feasibility Analysis}
Feasibility analysis aims to uncover the strengths and weaknesses of 
a project. In its simplest term, the two criteria to judge feasibility 
are cost required and value to be attained. As such, a well-designed 
feasibility analysis should provide a historical background of the 
project, description of the project or service, details of the 
operations and management and legal requirements. Generally, feasibility 
analysis precedes technical development and project implementation. 
There is some feasibility factors by which we can determine that 
project is feasible or not:
\begin{itemize}
\item {\bf{Technical feasibility}}: Technological feasibility is carried 
out to determine whether the project has the capability, in terms of 
software, hardware, personnel to handle and fulfill the user 
requirements. The assessment is based on an outline design of system 
requirements in terms of Input, Processes, Output and Procedures. OFC
Automation Software is technically feasible as it is built up in Open 
Source Environment and thus it can be run on any Open Source platform.
\item {\bf{Economic feasibility}}: Economic analysis is the most 
frequently used method to determine the cost/benefit factor for 
evaluating the effectiveness of a new system. In this analysis we 
determine whether the benefit is gained according to the cost invested 
to develop the project or not. If benefits outweigh costs, only then 
the decision is made to design and implement the system. It is 
important to identify cost and benefit factors, which can be categorized 
as follows:
\begin{enumerate}
\item Development costs.
\item Operating costs.
\end{enumerate}
OFC Automation Software is also Economically feasible with 0 Development 
and Operating Charges as it is developed in Django framework and python 
language which is FOSS technology and the software is operated on Open 
Source platform.
\item {\bf {Legal feasibility}}: In this type of feasibility study, we 
basically determine whether the project conflicts with legal 
requirements, e.g. a data processing system must comply with the local 
Data Protection Acts. But OFC Automation Software has been developed 
for the Office Automation process with properly Licensed technologies. 
Thus is the legal process.
\item {\bf{Operational feasibility}}: Operational feasibility is a measure 
of how well a project solves the problems, and takes advantage of the 
opportunities identified during scope definition and how it satisfies 
the requirements identified in the requirements analysis phase of system 
development. All the operations performed in the software are very quick 
and satisfy all the reuirements.
\item {\bf{Behaviour Feasibility}}: In this feasibility, we check about the 
behavior of the proposed system software i.e. whether the proposed 
project is user friendly or not, whether users can use the project 
without any training because of the user friendliness or not. OFC
Automation Software is very user friendly as its users interact with it 
through web.
\end{itemize}

\section{Software Requirement Analysis}
A Software Requirements Analysis for a software system is a complete 
description of the behaviour of a system to be developed. It includes 
a set of use cases that describe all the interactions the users will 
have with the software. In addition to use cases, the SRS also contains 
non-functional requirements. Non-functional requirements are 
requirements which impose constraints on the design or implementation.
\begin{itemize}
\item{\bf Purpose}: OFC Automation Software is a web based software and the 
main purpose of this project is to:
\begin{enumerate}
\item Perform most of the task of Open Freight Carriers online 
and make it dynamic.
\item Make the Registration and Searching easier.
\item Automatic calculation of the amount for the work done.
\item Reduce the dependencies between people involved in the process.
\item Increasing the transparency.
\end{enumerate}
\item{\bf General Description}: OFC Automation Software is basically 
designed for those companies or Organisations which provide different 
types of services to all types of clients. Keeping track of different 
works done by different clients and then getting all the reports of 
the work done is not an easy job. To make these tasks easy with all 
functions performed quickly, Automation Software will be quiet helpful.

Administrator will be the super user of the application who will 
configure system information such as adding new products and there 
information or editing or deleting the old ones, managing employees 
and clients.

It will be an enterprise software, so it is distributed and data centric. 
This Software is designed on the basis of web application architecture. 
In this application, MySQL database will be used to store data related 
to employees, material, jobs, labs, tests, clients, amounts etc. Since 
database is on Server, so any number of users can work simultaneously 
and can share their data with each other. It is developed using Django, 
Python, HTML, CSS and JavaScript.
\item{\bf Users of the System}
\begin{enumerate} 
\item Administrator : Administrator can add or update 
(activate/inactivate) the details, and also can see information of all 
employees and can see his or her information. New labs, materials or 
tests can be added or the existing can also be updated.
\item Employee : As employees are directly related to clients, so they 
are able to add or update the details of clients using this section. 
Administrator can see all the clients. Employees can manage their 
clients only, and particular client can see his or her detail.
\item Client : Clients are the end users that benefit from the 
Automation Software. A client can get information of all services 
available, and thus can apply for same. They can also view the status 
of the number of the previous jobs done by them in the Organisation.
\end{enumerate}
\end{itemize}
\subsection{Functional Requiremets}
\begin{itemize}
\item {\bf Specific Requirements}: This phase covers the whole requirements 
for the system. After understanding the system we need the input data 
to the system then we watch the output and determine whether the output 
from the system is according to our requirements or not. So what we have 
to input and then what we’ll get as output is given in this phase. This 
phase also describe the software and non-function requirements of the 
system.
\item {\bf Input Requirements of the System}
\begin{enumerate} 
\item Client Details
\item Job Details
\item Extra Charges Details
\item Lab Details
\item Organisation \& Department Details
\item Rate List
\item Staff Details
\end{enumerate}
\vskip 0.5cm
\item {\bf Output Requirements of the System}
\begin{enumerate} 
\item Interface for administrator to configure the system.
\item Listing of all the services offered.
\item Interface for clients and employees.
\item Automatic generation of Reports, Bills, Receipts, and Vouchers 
for clients.
\item Calculation of Job amount.
\item Generation of Registers with Certain requirements.
\end{enumerate}
\vskip 0.5cm
\item {\bf Special User Requirements}
\begin{enumerate} 
\item Automatic Email Generation and Sending to the concerned person.
\end{enumerate}
\vskip 0.5cm
\item {\bf Software Requirements}
\begin{enumerate} 
\item Programming language: Python 2.7
\item Framework: Django 1.4 
\item Web Languages: Html, Java Script, CSS 
\item Database: MySQL Database Server 5.1 
\item Documentation: Doxygen 1.8.3
\item Text Editor: Gedit, Geany, Vim
\item Operating System: Ubuntu 12.04 or up
\item Debugger: Django Debugger, Django shell, Terminal
\item Web Server: Apache 2.4
\end{enumerate}
\vskip 0.5cm
\subsection{Non functional requirements}
\begin{enumerate} 
\item Scalability: System should be able to handle a number of users. 
For e.g., handling around thousand users at the same time.
\item Usability: Simple user interfaces that a layman can understand.
\item Speed: Speed of the system should be responsive i.e. Response to
 a particular action should be available in short period of time. For 
e.g., Updating the project tasks take few seconds for the changes if 
the entry is not starred.
\end{enumerate}
\end{itemize}
