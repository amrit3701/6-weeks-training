\section{Overview}
Structure Information Modeling (SIM) is an open-source, free web-based software, developed by students
of Testing and Consultancy Cell (TCC), under the guidance of Dr. H.S. Rai.The software to be designed will create an index model for a 3 dimensional civil structure like a building or any other civil engineering project. This requires a software for designing such
structures using a computer, the one used in SIM is STAAD PRO. In STAAD PRO the engineer
draws the structure using various graphical tools and accordingly a .STD file is created that has
all information about the building structure. Alternatively the engineer can choose to work from
this file only and creating the elements of the building by writing pre-defined instructions into the
file just like a script. It is this file that serves as the fundamental requirement for SIM. SIM parses
this file and stores the parsed information into a database which is of object orientation nature.

SIM will serve multiple users at a time. This is achieved by providing a web interface to things
so that various user can access it from their web browsers. The user is required to make a decision
at the home page of whether to upload a new file or open an already uploaded file. If the user
chooses the former then the user must provide with a valid and complete STAAD PRO file for the
system to further process.

Also in this project, the user makes the drawings of different views of the building on the drawing sheets. The user
will be able to enter the specifications of the building through the web browser and on
the back-end, the FreeCAD macros will use those input values to draw the drawings of different
views of the building on different drawing sheets. The output of the same can then be taken by a
user in SVG, PDF and fcstd formats. The user will also facilitate to render the 3D model of the building on the browser screen.



%The  term  computer  graphics  includes  almost  everything  on  computers  that  is  not  text  or  sound.  Today
%almost  every  computer  can  do  some  graphics,  and  people  have  even  come  to  expect  to  control  their
%computer  through  icons  and  pictures  rather  than  just  by  typing.  Here  in  our  lab  at  the  Program  of
%Computer  Graphics,  we  think  of  computer  graphics  as  drawing  pictures  on  computers,  also  called
%rendering.  The  pictures  can  be photographs, drawings, movies, or simulations ­­ pictures of things which
%do  not  yet  exist  and  maybe  could   never  exist.  Or  they  may  be  pictures  from  places  we  cannot  see
%directly,  such  as  medical  images  from  inside  your  body.  Computer  graphics  is  now  used  in  various
%fields;  for  industrial,  educational,  medical  and  entertainment  purposes.  The  aim  of  computer  graphics  is
%to  visualize  real  objects  and  imaginary  or  other  abstract  items.  In  order to visualize various things, many
%technologies  are  necessary  and  they  are  mainly  divided  into  two  types  in  computer  graphics:  modeling
%and  rendering  technologies.\\

\section{The Existing System}
There are few existing systems for solving this particular problem like STAAD.Pro, SAP2000 but
they don’t have following features required by our mentor. These system were not open source
and free web based software that were need.

All exiting system suffers from at least one of the following system.\\
\textbf{Limitations of previous system}
\begin{itemize}
\item No batch mode
\item They are costly (STAAD.Pro costs nearly 80,000 rupees)
\item Not open-source (User not modified the software)
\item They need installation and a lot of system resources
\item Platform dependent
\item There is no any automations of drawing of 3D object (step-by-step approach is following for making drawings)
\item Difficulty for taking sectional view of the object
\item Doesn't support WebGl and IFC format
\end{itemize}

\section{User Requirement Analysis}
For User Requirement Analysis, users of this system have been asked about possible requirements
that this software should have and we got following resultant list of outputs-:

\begin{enumerate}
\item Provide on-line way to analysis so that individual does not have to install anything.
\item Make it work like batch mode. so, that user can give inputs together and relax.
\item Help M.Tech and Civil Engineer to analysis structure.
\item Storing Staad Pro file to the database so that read data through queries.
\item All the data of Staad Pro file is store in database like a tree. By the result it is very easy to read data.
\item Rendering 3D model on browser without any plugin installation. 
\item Make PDF of different views of the building 
\end{enumerate}

\section{Feasibility Analysis}
Feasibility analysis aims to uncover the strengths and weaknesses of a project. In its simplest term,
the two criteria to judge feasibility are cost required and value to be attained. As such, a well-
designed feasibility analysis should provide a historical background of the project, description of the
project or service, details of the operations and management and legal requirements. Generally,
feasibility analysis precedes technical development and project implementation. There is some
feasibility factors by which we can determine that project is feasible or not:

\begin{itemize}
\item \textbf{Technical feasibility}: Technological feasibility is carried out to determine whether the
project has the capability, in terms of software, hardware, personnel to handle and fulfill the
user requirements. This whole project is based on parsing Staad Pro file to the database and 
on the FreeCAD and Django for user interface. Technical feasibility of this project revolves around
the technical boundaries and limitations of the FreeCAD and Django. Structure Information Modeling (SIM) is technically feasible as it is built up in Open Source
Environment and thus it can be run on any Open Source platform. 

\item \textbf{Economic feasibility}: Economic analysis is the most frequently used method to determine
the cost/benefit factor for evaluating the effectiveness of a new system. In this analysis we
determine whether the benefit is gain according to the cost invested to develop the project
or not. If benefits outweigh costs, only then the decision is made to design and implement
the system. It is important to identify cost and benefit factors, which can be categorized as
follows:

\begin{enumerate}
\item Development costs.
\item Operating costs.
\end{enumerate}

Structure Information Modeling Software is also Economically feasible with 0 Development and Op-
erating Charges as it is developed in Django framework and FreeCAD which is
FOSS technology and the software is operated on Open Source platform.

\item \textbf{Operational feasibility}: Operational feasibility is a measure of how well a project solves
the problems, and takes advantage of the opportunities identified during scope definition
and how it satisfies the requirements identified in the requirements analysis phase of system
development. All the Operations performed in the software are very quick and satisfies all
the reuirements. This project is also operational feasible as it automates the work of solving
the problem of analysising the structures which not only saves time but also saves money as
most of the work is done by Employees and M.Tech students is done by this software.

\section{Objective of Project}
Structure Information Modeling Software is a web based software (that means it can run on any operation system) and the main objectives of this project is to:
\begin{enumerate}
\item To inspire M.Tech students to automate their work and do programming
\item Perform most of difficult Calculation work
\item Make it work like batch mode. so, that user can give inputs together and relax.
\item Accept inputs from the user in *.std file format
\item Help M.Tech and Civil Engineer to analysis structure.
\item Reduce the time for analysis.
\item Generates the final output in the form of PDF, SVG and .fcstd
\item Provide on-line way to analysis so that individual does not have to install anything.
\item Adminstration login page
\end{enumerate}

\end{itemize}

\begin{comment}
\section{Introduction to STAAD Pro}
STAAD.Pro is a general purpose structural analysis and design program with applications primarily in the building industry - commercial buildings, bridges and highway structures, industrial structures, chemical plant structures, dams, retaining walls, turbine foundations, culverts and other embedded structures, etc. The program hence consists of the following facilities to enable this task.

\begin{itemize}
\item Graphical model generation utilities as well as text editor based commands for creating the mathematical model. Beam and column members are represented using lines. Walls, slabs and panel type entities are represented using triangular and quadrilateral finite elements. Solid blocks are represented using brick elements. These utilities allow the user to create the geometry, assign properties, orient cross sections as desired, assign materials like steel, concrete, timber, aluminum, specify supports, apply loads explicitly as well as have the program generate loads, design parameters etc. 

\item Analysis engines for performing linear elastic and pdelta analysis, finite element analysis, frequency extraction, and dynamic response (spectrum, time history, steady state, etc.).

\item  Design engines for code checking and optimization of steel, aluminum and timber members. Reinforcement calculations for concrete beams, columns, slabs and shear walls. Design of shear and moment connections for steel members. 

\item Result viewing, result verification and report generation tools for examining displacement diagrams, bending moment and shear force diagrams, beam, plate and solid stress contours, etc. 

\item Peripheral tools for activities like import and export of data from and to other widely accepted formats, links with other popular softwares for niche areas like reinforced and prestressed concrete slab design, footing design, steel connection design, etc. 

\item A library of exposed functions called OpenSTAAD which allows users to access STAAD.Pro’s internal functions and routines as well as its graphical commands to tap into STAAD’s database and link input and output data to third-party software written using languages like C, C++, VB, VBA, FORTRAN, Java, Delphi, etc. Thus, OpenSTAAD allows users to link in-house or third-party applications with STAAD Pro.
\end{itemize}

STAAD Pro is basically based on Finite Element Analysis and is programmed in C for carrying out the computations for Analysis and Design of a Structure. STAAD Pro works in a very user friendly environment with a option of shifting between graphical and analytical modes any time between the work.

With its features it is one of the most common softwares used in industry.
\section{Introduction to CAD}

Computer-aided design (CAD) is the use of computer systems to assist in the creation, modification, analysis, or optimization of a design. CAD software is used to increase the productivity of the designer, improve the quality of design, improve communications through documentation, and to create a database for manufacturing.CAD is an important industrial art extensively used in many applications, including automotive, shipbuilding, and aerospace industries, industrial and architectural design, prosthetics, and many more. CAD is also widely used to produce computer animation for special effects in movies, advertising and technical manuals. CAD output is often in the form of electronic files for print, machining, or other manufacturing operations. CAD is also used for the accurate creation of photo simulations that are often required in the preparation of Environmental Impact Reports, in which computer-aided designs of intended buildings are superimposed into photographs of existing environments to represent what that locale will be like were the proposed facilities allowed to be built.  Computer­ Aided  Drafting  describes  the  process  of  creating  a
technical  drawings with the use of computer software. CAD software is used to increase the productivity
of  the  designer,  improve  the  quality  of  design,  improve  communications  through  documentation,  and  to
create  a  database  for  manufacturing.  CAD  output  is  often  in  the  form  of  electronic  files  for  print  or
machining  operations.  CAD  software  uses  either  vector  based  graphics  to  depict  the  objects  of
traditional  drafting,  or  may  also  produce  raster  graphics  showing  the  overall   appearance  of  designed
objects.\\

  Today  there  are  very  few  aspects  of  our  lives  not  affected  by  computers.  Practically  every  cash or
monetary  transaction  that  takes  place  daily  involves  a  computer.  In  many  cases,  the  same  is  true  of
computer  graphics.  Whether  you  see  them  on  television,  in  newspapers,  in  weather   reports  or  while  at
the  doctor’s  surgery,  computer  images  are  all  around  you.  “A  picture  is  worth  a  thousand  words”  is  a
well ­known  saying  and  highlights  the  advantages  and  benefits  of  the  visual ation of our data. We
are  able  to  obtain  a  comprehensive  overall  view  of  our  data  and   also  study  features  and  areas  of
particular  interest. A  range  of  tools
and  facilities  are  available  to  enable  users  to  visualize  their  data,  and  this  document  provides  a  brief
summary  and  overview.  Computer  graphics  can  be  used  in  many  disciplines.  Charting,  ations,
Drawing,  Painting  and  Design,  Image  Processing  and  Scientific  Visualization  are  some  among  them.
Computer  graphics  is concerned with all aspects of producing images using a computer. It concerns with
the pictorial synthesis of real or imaginary objects from their computer­based models.\\

  CAD  often  involves  more  than  just  shapes.  As  in  the  manual  drafting  of  Technical  and  Engineering
Drawings,  the  output  of  CAD  must  convey  information,  such  as  material,  processes,dimensions  and
tolerances,  according  to  application­specific  conventions.  CAD  may  be  used  to  design  curves  and
figures  in  two­Dimensional(2D)  space;  or  curves,  surfaces,  and  solids  in  three  dimensional(3D)
space.CAD  is  an  important  industrial  art  extensively  used  in  many  applications,  including  automotive,
shipbuilding,  and  aerospace  industries,  industrial  and  architectural  design, prosthetic,  and  many  more.
CAD  is  also  widely  used  to  produce  Computer  animation  for  special  Effects  in  movies,advertising  and
technical  manuals.  The  modern  ubiquity  and  power  of  computers  means  that  even  perfume  bottles  and
shampoo  dispensers  are  designed  using   techniques unheard of by engineers of the 1960s. Because  of its
enormous  economic  importance,  CAD  has  been  a  major  driving  force  for  research  in  computational
geometry,  computer  graphics  (both  hardware  and  software),  and  discrete  differential  geometry.The
design  of  geometric  model  for  object  shapes,  in  particular,  is  occasionally  called  Computer ­Aided
Geometric  Design  (CAGD).While  the  goal  of  automated  CAD  systems  is to increase efficiency, they are
not  necessarily  the  best  way   to  allow  newcomers  to  understand  the  geometrical  principles  of  Solid
Modeling.\\

  I  explored  FreeCAD's  source  code.  FreeCAD is Free and Open Source CAD Software. FreeCAD is
a  fully  comprehensive  3D  CAD   application  that  you  can  download  and  install  for  free.  There  is  a  large
base  of  satisfied  FreeCAD   users  worldwide,  and  it  is  available  in  more  than  20  languages  and  for  all
major  operating  systems,  including  Microsoft  Windows,  Mac  OS  X  and  Linux  (Debian,  Ubuntu,
Fedora,  Mandriva,  Suse  ...).  FreeCAD  is  an  application  for Computer  Aided  Design  (CAD)  in  three
dimensions  (2d).  With  FreeCAD   you   can  create  technical  drawings  such  as  plans  for  buildings,  interiors,
mechanical parts or schematics and diagrams.\\

\end{comment}
\begin{comment}
The  app  is  great  for  industrial  designers,  but  anyone  who  wants  to  learn  how  to  make  2D  CAD
drawings  will  like  this  program.  For  a  free  software,  LibreCAD  gives  you  a  lot  of  tools  to  work  with.
New  users  will  be  able  to  create  basic drawings, while advanced users can make engineering plans with
5the  software.  Layers  can  be  added,  ideal  for  complex   drawings.  The  provided  tools  are  sufficient  for
producing  high  precision  drawings.  You  can  start  drawings  from  scratch.  But  it  is  also  easy  to  put  in
splines,  ellipses,  arcs,  lines  and   circles.  A  single  item  can  have  several iterations. For instance, you have
4 modes for a rectangle parameter. The different shapes can be combined easily.
LibreCAD  also  has  a  powerful  zoom  tool  that  lets  you  look  at  models  at  different  distances.  This  is
essential  for  designers  who  are  going  to  make  life­size  copies  of  a  drawing.  There are three tabs above
the  working  area.   The  first  tab  is  for  changing  color,  useful  for  layer  definition.   The  other  tab  is  for
changing size and the third for workspace customization.\\

LibreCAD also has grids which are extremely useful for those new to CAD. Once you have  made the
basic  object,  you can customize it in many ways. Scaling is particularly easy here. Also worth mentioning
here  is  the  "Explode  text  into  letters"  effect.   It  is  a  special  feature  that  will come in handy ations.
LibreCAD  allows  you  to  put  horizontal  or vertical restrictions on completed models. Relative zeros may
be  locked,  useful  for   ending  and  starting  points.  All  in  all,  it is powerful, free CAD application. You can
download, install and distribute LibreCAD freely, with no fear of copyright infringement.
\end{comment}
