\section{Inkscape}
Inkscape is an open source drawing tool for creating and editing SVG graphics. More than just a text vector
editor, Inkscape provides a WYSIWYG interface for manipulation of vector images, allowing the artist to
express himself freely. While other free and proprietary software exists with similar capabilities, Inkscape
provides an interface to directly manipulate the underlying SVG code, which allows one to be certain that the
code is in compliance with W3C standards. Since the beginning of its development, the Inkscape project has
been a very active, providing stability for the current software and growth of capacities in the future.
Like other drawing programs, Inkscape offers creation of basic shapes (such as ellipses, rectangles, stars,
polygons and spirals) as well as the ability to transform and manipulate these basic shapes by rotation,
stretching and skewing.

Inkscape also offers functionality to manipulate objects more precisely by adjusting node points and curves.
These functions are indispensable to useful drawing software, and allow the advanced artist to freely create
what he imagines.

The properties of objects can either be manipulated individually and precisely through the XML editor, or
more generally in an intuitive fashion by input devices such as mice, pen tablets or even touch screen.
In addition, Inkscape allows one to insert text and bitmaps (such as PNG, another W3C recommended bitmap
image format) into an image as well as perform some basic editing functions on them. If further bitmap
editing is required, other tools may be used (such as the GIMP) on images before importing them or after. In
fact, if a linked bitmap is edited in another program, Inkscape will reflect these changes once the SVG is
reloaded.

All of these characteristics make Inkscape a model drawing application, especially considering its flexibility
and many other capabilities. Its strict compliance with the W3C SVG standards allow excellent portability of
images to many applications and platforms on which these applications are used.

\subsection{About SVG}
Those who work with graphics for internet use are familiar with the problems tied to publication of images on
the web. Traditionally, bitmap images (such as JPG or GIF) have been the only option for use in such
documents, with the disadvantage that these images are either too large for quick transfer or, if they are small
or highly compressed to reduce file-size, of poor quality.

As a solution to this problem, Macromedia created the Flash image format. While Flash satisfactorily solved
the main problems inherent to bitmap images, there has been discontent for some users that the common
vector format for the web is dependent solely on Macromedia for development of the file format and
software. In order to address this discontent and provide an open option for vector graphics, the W3C created
the SVG file format, making a freely usable vector format available to everyone.

Most image files are only able to be read by specific software that renders the image. SVG, however, is
described in XML and CSS, and its files can be opened and edited in any ASCII text editor. While it is
possible to create SVG images in this manner, it is highly unproductive and unintuitive. SVG editors and
renderers have the ability to easily open and manipulate SVG files without a special interpreter.

\subsection{Objectives of the SVG Format}
The advantages of SVG are the same as for any vector image: high-quality images that are smooth and crisp
ability to resize the image to any dimensions without diminishing quality, which is impossible with bitmap
images. The SVG standard also defines animation, and with a little use of Javascript, one can make SVG
interactive. Finally, since SVG is written in XML, it is possible to create graphics based on data that is stored
in other XML-based formats, such as graphs, charts and maps. Despite its benefits, there is a lack of usable
software to create and edit SVG files and take full advantage of its capacities; for this reason, SVG is not as
usable at the moment as Flash.

\subsection{Export PDF macro}
It uses Inkscape to do the SVG to PDF conversion. Customize ToolsBar 
instructions can be used to add tool button in Drawing WB and therefore 
to be able to use Export PDF macro with ease when needed.
\begin{lstlisting}  
import os
import subprocess
from PySide import QtGui

obj = FreeCADGui.Selection.getSelection()[0]

# Path to Inkscape executable
inkscape_path = "/usr/bin/inkscape"
# Exported PDF file location
file_location = os.path.expanduser("~" + os.sep + obj.Label + ".pdf")

# -------------------------------------------------------------------
try:
  page = getattr(obj, 'PageResult')
  except AttributeError:
    QtGui.QMessageBox.warning(None,"Export PDF", "Page object has \
        not been selected.")
    else:
        file_exists = os.path.isfile(file_location)
            if file_exists == True:
                QtGui.QMessageBox.warning(None,"Export PDF", "A \
                    file with the name " + obj.Label + ".pdf already \
                        exists:\n\n" + file_location)
                    else:
                         call_inkscape = [inkscape_path, "-f", \
                            obj.PageResult,"-A", file_location]
                         subprocess.call(call_inkscape)
                         QtGui.QMessageBox.information(None,"Export \
                            PDF", "Successfully exported:\n\n" + \
                                file_location)
\end{lstlisting}  

\subsubsection{What does it do?}
It exports drawing page to PDF file using Inkscape. Instead of bitmap 
image inserted in PDF file elements like text are searchable and exported 
PDF file size can be smaller when the drawings do not have big number of 
elements in it.
\subsubsection{How to use it}
\begin{itemize}
\item Select Page object in Tree View.
\item Run Export PDF macro.
\end{itemize}
