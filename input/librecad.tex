\begin{comment}

\subsection{Introduction to FreeCAD}

FreeCAD is a general purpose parametric 3D CAD modeler. The development 
is completely Open Source (LGPL License). FreeCAD is aimed directly at 
mechanical engineering and product design but also fits in a wider range 
of uses around engineering, such as architecture or other engineering specialties.

FreeCAD features tools similar to Catia, SolidWorks or Solid Edge, and 
therefore also falls into the category of MCAD, PLM, CAx and CAE. It is a 
feature based parametric modeler with a modular software architecture which 
makes it easy to provide additional functionality without modifying the core system.

As with many modern 3D CAD modelers it has many 2D components in order 
to sketch 2D shapes or extract design details from the 3D model to create 
2D production drawings, but direct 2D drawing (like AutoCAD LT) is not the 
focus, neither are animation or organic shapes (like Maya, 3ds Max, Blender 
or Cinema 4D), although, thanks to its wide adaptability, FreeCAD might 
become useful in a much broader area than its current focus.

FreeCAD makes heavy use of all the great open-source libraries that exist 
out there in the field of Scientific Computing. Among them are OpenCascade, 
a powerful CAD kernel, Coin3D, an incarnation of Open Inventor, Qt, the world-famous 
UI framework, and Python, one of the best scripting languages available. 
FreeCAD itself can also be used as a library by other programs.

FreeCAD is also fully multi-platform, and currently runs flawlessly on 
Windows and Linux/Unix and Mac OSX systems, with the exact same look and 
functionality on all platforms.

\subsection{About FreeCAD}
FreeCAD is maintained and developed by a community of enthusiastic developers 
and users (see the contributors page). They work on FreeCAD voluntarily, 
in their free time. They cannot guarantee that FreeCAD contains or will 
contain everything you might wish, but they will usually do their best! 
The community gathers on the FreeCAD forum, where most of the ideas and 
decisions are discussed.

%----------------------------------------------------------------------------------------
\
\
\
%-------------------------------------------------------------------

\subsection{FreeCAD's features}
\subsubsection{Key features}
\begin{itemize}
\item A complete Open CASCADE Technology-based geometry kernel allowing 
complex 3D operations on complex shape types, with native support for 
concepts like brep, nurbs curves and surfaces, a wide range of geometric 
entities, boolean operations and fillets, and built-in support of STEP and 
IGES formats.
\item A full parametric model. All FreeCAD objects are natively parametric, 
which means their shape can be based on properties or even depend on other 
objects, all changes being recalculated on demand, and recorded by the 
undo/redo stack. New object types can be added easily, that can even be 
fully programmed in Python.
\item A modular architecture that allow plugins (modules) to add functionality
to the core application. Those extensions can be as complex as whole new
applications programmed in C++ or as simple as Python scripts or self-recorded
macros. You have complete access from the Python built-in interpreter,
macros or external scripts to almost any part of FreeCAD, being geometry 
creation and transformation, the 2D or 3D representation of that geometry 
(scenegraph) or even the FreeCAD interface.
\item Import/export to standard formats such as STEP, IGES, OBJ, STL, DXF, 
SVG, STL, DAE, IFC or OFF, NASTRAN, VRML in addition to FreeCAD's native 
Fcstd file format. The level of compatibility between FreeCAD and a given 
file format can vary, since it depends on the module that implements it.
\item A Sketcher with constraint-solver, allowing to sketch geometry-constrained 
2D shapes. The sketcher currently allows you to build several types of 
constrained geomerty, and use them as a base to build other objects throughout 
FreeCAD.
\item A Robot simulation module that allows to study robot movements. The 
robot module already has an extended graphical interface allowing GUI-only 
workflow.
\item A Drawing sheets module that permit to put 2D views of your 3D 
models on a sheet. This modules then produces ready-to-export SVG or PDF 
sheets. The module is still sparse but already features a powerful Python 
functionality.
\item A Rendering module that can export 3D objects for rendering with 
external renderers. Currently only supports povray and LuxRender, but is 
expected to be extended to other renderers in the future.
\item An Architecture module that allows BIM-like workflow, with IFC 
compatibility. The making of the Arch module is heavily discussed by the 
community here.
\end{itemize}

\subsubsection{Genral features}
\begin{itemize}
\item \textbf{FreeCAD is multi-platform.} It runs and behaves exactly the same 
way on Windows Linux and Mac OSX platforms.
\item \textbf{FreeCAD is a full GUI application.} FreeCAD has a complete Graphical 
User Interface based on the famous Qt framework, with a 3D viewer based on
Open Inventor, allowing fast rendering of 3D scenes and a very accessible 
scene graph representation.
\item \textbf{FreeCAD also runs as a command line application,} with low memory 
footprint. In command line mode, FreeCAD runs without its interface, but 
with all its geometry tools. It can be, for example, used as server to 
produce content for other applications.
\item \textbf{FreeCAD can be imported as a Python module,} inside other applications 
that can run python scripts, or in a python console. Like in console mode, 
the interface part of FreeCAD is unavailable, but all geometry tools are accessible.
\item \textbf{Workbench concept:} In the FreeCAD interface, tools are grouped 
by workbenches. This allows to display only the tools used to accomplish 
a certain task, keeping the workspace uncluttered and responsive, and the 
application fast to load.
\item \textbf{Plugin/Module framework for late loading of features/data-types.} 
FreeCAD is divided into a core application and modules, that are loaded 
only when needed. Almost all the tools and geometry types are stored in 
modules. Modules behave like plugins, and can be added or removed to an 
existing installation of FreeCAD.
\item \textbf{Parametric associative document objects:} All objects in a FreeCAD 
document can be defined by parameters. Those parameters can be modified 
on the fly, and recomputed anytime. The relationship between objects is 
also stored, so modifying one object also modifies its dependent objects.
\item \textbf{Parametric primitive creation} (box, sphere, cylinder, etc).
\item Graphical \textbf{modification operations} like translation, rotation, 
scaling, mirroring, offset (trivial or after Jung/Shin/Choi) or shape 
conversion, in any plane of the 3D space.
\item \textbf{Boolean operations} (union, difference, intersect)
\item Graphical creation of \textbf{simple planar geometry} like lines, wires, 
rectangles, arcs or circles in any plane of the 3D space.
\item Modeling with straight or revolution \textbf{extrusions, sections} and 
\textbf{fillets}.
\item Topological components like \textbf{vertices, edges, wires} and \textbf{planes} 
(via python scripting).
\item \textbf{Testing and repairing} tools for meshes: solid test, non-two-manifolds 
test, self-intersection test, hole filling and uniform orientation.
\item \textbf{Annotations} like texts or dimensions
\item \textbf{Undo/Redo framework:} Everything is undo/redoable, with access to 
the undo stack, so multiple steps can be undone at a time.
\item \textbf{Transaction management:} The undo/redo stack stores document transactions 
and not single actions, allowing each tool to define exactly what must be 
undone or redone.
\item \textbf{Built-in scripting framework:} FreeCAD features a built-in Python 
interpreter, and an API that covers almost any part of the application, 
the interface, the geometry and the representation of this geometry in the 
3D viewer. The interpreter can run single commands up to complex scripts, 
in fact entire modules can even be programmed completely in Python.
\item \textbf{Built-in Python console} with syntax highlighting, autocomplete 
and class browser: Python commands can be issued directly in FreeCAD and 
immediately return results, permitting scriptwriters to test functionality 
on the fly, explore the contents of the modules and easily learn about 
FreeCAD internals.
\item \textbf{User interaction mirroring on the console:} Everything the user 
does in the FreeCAD interface executes python code, which can be printed 
on the console and recorded in macros.
\item \textbf{Full macro recording and editing:} The python commands issued when 
the user manipulates the interface can then be recorded, edited if needed, 
and saved to be reproduced later.
\item \textbf{Compound (ZIP based) document save format:} FreeCAD documents 
saved with .fcstd extension can contain many different types of information, 
such as geometry, scripts or thumbnail icons. The .fcstd file is itself a 
zip container, so a saved FreeCAD file has already been compressed.
\item \textbf{Fully customizable/scriptable Graphical User Interface.} The Qt-based 
interface of FreeCAD is entirely accessible via the python interpreter. 
Aside from the simple functions that FreeCAD itself provides to workbenches, 
the whole Qt framework is accessible too, allowing any operation on the GUI, 
such as creating, adding, docking, modifying or removing widgets and toolbars.
\item \textbf{Thumbnailer} (Linux systems only at the moment): The FreeCAD 
document icons show the contents of the file in most file manager applications 
such as gnome's nautilus.
\item \textbf{A modular MSI installer} allows flexible installations on Windows 
systems. Packages for Ubuntu systems are also maintained.
\end{itemize}

%----------------------------------------------------------------------
\
\
\
%----------------------------------------------------------------------

\subsection{Application and User Interface}
Like almost everything else in FreeCAD, the user interface part (Gui) is 
separated from the base application part (App). This is also valid for 
documents. The documents are also made of two parts: the Application document, 
which contains our objects, and the View document, which contains the 
representation on screen of our objects.

Think of it as two spaces, where the objects are defined. Their constructive 
parameters (is it a cube? a cone? which size?) are stored in the Application 
document, while their graphical representation (is it drawn with black lines? 
with blue faces?) are stored in the View document. Why is that? Because 
FreeCAD can also be used WITHOUT graphical interface, for example inside 
other programs, and we must still be able to manipulate our objects, even 
if nothing is drawn on the screen.

Another thing that is contained inside the View document are 3D views. 
One document can have several views opened, so you can inspect your document 
from several points of view at the same time. Maybe you would want to see a 
top view and a front view of your work at the same time? Then, you will have 
two views of the same document, both stored in the View document. Creating 
new views or closing views can be done from the View menu or by right-clicking 
on a view tab.

%----------------------------------------------------------------------------------------
\
\
\
%-------------------------------------------------------------------

\subsection{How it started?}

%--------------
\
\
%--------------------
FreeCAD's focus is to allow you to make high-precision 3D models, to keep 
tight control over those models (being able to go back into modelling history 
and change parameters), and eventually to build those models (via 3D printing, 
CNC machining or even construction worksite). It is therefore very different 
from some other 3D applications made for other purposes, such as animation 
film or gaming. Its learning curve can be steep, specially if this is 
your first contact with 3D modeling. If you are struck at some point, don't 
forget that the friendly community of users on the FreeCAD forum might be 
able to get you out in no time.

The workbench you will start using in FreeCAD depends on the type of job 
you need to do: If you are going to work on mechanical models, or more 
generally any small-scale objects, you'll probably want to try the PartDesign 
Workbench. If you will work in 2D, then switch to the Draft Workbench, or the 
Sketcher Workbench if you need constraints. If you want to do BIM, launch 
the Arch Workbench. If you are working with ship design, there is a special 
Ship Workbench for you. And if you come from the OpenSCAD world, try 
the OpenSCAD Workbench.

You can switch workbenches at any time, and also customize your favorite 
workbench to add tools from other workbenches.


\subsection{Downloading Source Code}
Fired up the terminal because you need to install the qt4 development libraries, tools, compiler and git.
\begin{verbatim} 
git clone https://github.com/FreeCAD/FreeCAD free-cad-code
\end{verbatim} 
Getting the dependencies:
\begin{verbatim}                                                      
sudo apt install build-essential cmake python python-matplotlib 
libtool libcoin80-dev libsoqt4-dev libxerces-c-dev libboost-dev 
libboost-filesystem-dev libboost-regex-dev libboost-program-options-dev 
libboost-signals-dev libboost-thread-dev libboost-python-dev 
libqt4-dev libqt4-opengl-dev qt4-dev-tools python-dev python-pyside 
pyside-tools oce-draw libeigen3-dev libqtwebkit-dev libshiboken-dev 
libpyside-dev libode-dev swig libzipios++-dev libfreetype6 
libfreetype6-dev liboce-foundation-dev liboce-modeling-dev 
liboce-ocaf-dev liboce-visualization-dev liboce-ocaf-lite-dev 
libsimage-dev checkinstall python-pivy python-qt4 doxygen libspnav-d              
\end{verbatim} 
Compiling:
Now first we’ll create a separate directory for storing our build files. 
This will be out-of-source build. Which means the source code directory 
will not get modified and there won’t be any issue with git.
\begin{verbatim} 
mkdir build
cd build
cmake ../free-cad-code
make
\end{verbatim} 
Execution:
Execute using the following command:
\begin{verbatim}                                                      
./bin/FreeCAD
\end{verbatim} 
\end{comment}                                                     
